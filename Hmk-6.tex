%%%%%%%%%%%%%%%%%%%%%%%%%%%%%%%%%%%%%%%%%
% Short Sectioned Assignment
% LaTeX Template
% Version 1.0 (5/5/12)
%
% This template has been downloaded from:
% http://www.LaTeXTemplates.com
%
% Original author:
% Frits Wenneker (http://www.howtotex.com)
%
% License:
% CC BY-NC-SA 3.0 (http://creativecommons.org/licenses/by-nc-sa/3.0/)
%
%%%%%%%%%%%%%%%%%%%%%%%%%%%%%%%%%%%%%%%%%

%----------------------------------------------------------------------------------------
%	PACKAGES AND OTHER DOCUMENT CONFIGURATIONS
%----------------------------------------------------------------------------------------

\documentclass[paper=a4, fontsize=11pt]{scrartcl} % A4 paper and 11pt font size

\usepackage[T1]{fontenc} % Use 8-bit encoding that has 256 glyphs
\usepackage{fourier} % Use the Adobe Utopia font for the document - comment this line to return to the LaTeX default
\usepackage[english]{babel} % English language/hyphenation
\usepackage{amsmath,amsfonts,amsthm} % Math packages

\usepackage{lipsum} % Used for inserting dummy 'Lorem ipsum' text into the template

\usepackage{sectsty} % Allows customizing section commands
\allsectionsfont{\centering \normalfont\scshape} % Make all sections centered, the default font and small caps

\usepackage{fancyhdr} % Custom headers and footers
\pagestyle{fancyplain} % Makes all pages in the document conform to the custom headers and footers
\fancyhead{} % No page header - if you want one, create it in the same way as the footers below
\fancyfoot[L]{} % Empty left footer
\fancyfoot[C]{} % Empty center footer
\fancyfoot[R]{\thepage} % Page numbering for right footer
\renewcommand{\headrulewidth}{0pt} % Remove header underlines
\renewcommand{\footrulewidth}{0pt} % Remove footer underlines
\setlength{\headheight}{13.6pt} % Customize the height of the header

\numberwithin{equation}{section} % Number equations within sections (i.e. 1.1, 1.2, 2.1, 2.2 instead of 1, 2, 3, 4)
\numberwithin{figure}{section} % Number figures within sections (i.e. 1.1, 1.2, 2.1, 2.2 instead of 1, 2, 3, 4)
\numberwithin{table}{section} % Number tables within sections (i.e. 1.1, 1.2, 2.1, 2.2 instead of 1, 2, 3, 4)

\setlength\parindent{0pt} % Removes all indentation from paragraphs - comment this line for an assignment with lots of text

%----------------------------------------------------------------------------------------
%	TITLE SECTION
%----------------------------------------------------------------------------------------

\newcommand{\horrule}[1]{\rule{\linewidth}{#1}} % Create horizontal rule command with 1 argument of height

\title{	
\normalfont \normalsize 
\textsc{University of Tennessee \\ Department of Mathematics} \\ [25pt] % Your university, school and/or department name(s)
\horrule{0.5pt} \\[0.4cm] % Thin top horizontal rule
\huge MATH 447 - Homework 6 \\ % The assignment title
\horrule{2pt} \\[0.5cm] % Thick bottom horizontal rule
}

\author{Robert D. French} % Your name

\date{\normalsize\today} % Today's date or a custom date

\begin{document}

\maketitle % Print the title

%----------------------------------------------------------------------------------------
%	PROBLEM 1
%----------------------------------------------------------------------------------------

\section*{Section 3.3, Ex. 3}
\newcommand{\nin}{n \in \mathbb{N}}
\newcommand{\pf}{\paragraph{Proof.}}
\newcommand{\done}{$\blacksquare$}

Let $x_1 \geq 2$ and $x_{n+1} := 1 + \sqrt{x_n - 1}$ for $\nin$. Show that $(x_n)$ is decreasing and bounded below by $2$. Find the limit.

\pf We can demonstrate by induction that $(x_n)$ is bounded below by 2. First note that $x_1 \geq 2$ by by hypothesis. Thus,

\begin{gather*}
x_1 \geq 2 \\
x_1 - 1 \geq 1\\
\sqrt{x_1 - 1} \geq 1\\
1 + \sqrt{x_1 - 1} \geq 1 + 1\\
x_2 \geq 2
\end{gather*}

so we have a basis for induction. Assume now that $x_n \geq 2$, and we will use this to demonstrate that $x_{n+1} \geq 2$:

\begin{gather*}
x_n \geq 2 \\
x_n - 1 \geq 1\\
\sqrt{x_n - 1} \geq 1\\
1 + \sqrt{x_n - 1} \geq 1 + 1\\
x_{n+1} \geq 2
\end{gather*}

Thus we have confirmed that $(x_n)$ is bounded below by 2.\\

We can also demonstrate by induction that $(x_n)$ is decreasing. Begin by noting that $x_1 \geq 2$, which implies that $x_1 - 1 \geq 1$. This tells us that $\sqrt{x_1} \geq 1$. Taking the difference of the previous equations tells us that $(x_1 - 1) - (\sqrt{x_1 - 1}) \geq 0$, and thus that $x_1 \geq 1 + \sqrt{x_1 - 1} = x_2$. Since $x_1 \geq x_2$, we have established a basis for induction.\\

Now assume that $x_{n-1} \geq x_n$ and recall that since 2 is a lower bound, $x_n - 1 \geq 1$. We aim to show that this implies $x_n \geq x_{n+1}$.

\begin{gather*}
x_{n-1} \geq x_n\\
x_{n-1} - 1 \geq x_n - 1\\
\sqrt{x_{n-1} - 1} \geq \sqrt{x_n - 1}\\
1 + \sqrt{x_{n-1} - 1} \geq 1 + \sqrt{x_n - 1}\\
x_n \geq x_{n+1}
\end{gather*}

Thus we have shown that $(x_n)$ decreases.\\

Since $(x_n)$ is decreasing, bounded below, and recursively defined, we may calculate its limit by acknowledging that $\lim(x_n) = \lim(x_{n+1})$, and for convenience we shall denote this $x*$. Thus,

\begin{gather*}
x^* = 1 + \sqrt{x^* - 1}\\
x^* + 1 = \sqrt{x^* - 1}\\
(x^* + 1)^2 = x^* - 1\\
{x^*}^2 - 3x^* + 2 = 0
\end{gather*}

The solutions of the final quadratic equation are 1 and 2, however, since 2 is a lower bound for $(x_n)$, we know its limit may not be 1. Thus, $\lim(x_n) = 2$.\done

\section*{Section 3.3, Ex. 4}

Let $x_1 := 1$ and $x_{n+1} := \sqrt{2 + x_n}$ for $\nin$. Show that $(x_n)$ converges and find the limit.

\pf $x_2 = \sqrt{3} < 2$, so we know that $x_1$ and $x_2$ are both less than 2. Assume $x_n < 2$, then

\begin{gather*}
x_n < 2\\
x_n + 2< 2 + 2\\
\sqrt{x_n + 2} < \sqrt{4}\\
x_{n+1} = \sqrt{x_n + 2} < 2\\
\end{gather*}

so by induction we have that $(x_n)$ is increasing and bounded above by 2. Since it is recursively defined, we may acknowledge that $\lim(x_n) = \lim(x_{n+1})$ and call that value $x^*$.

\begin{gather*}
x^* = \sqrt{2 + x^*}\\
(x^*)^2 = 2 + x*\\
(x^*)^2 - x* - 2 = 0\\
\end{gather*}

so that solving with the quadratic formula gives us $x^* = 2$. \done

\section*{Section 3.3., Ex. 5}

Let $y_1 := \sqrt{p}$, where $p > 0$, and $y_{n+1} := \sqrt{ p + y_n}$ for $\nin$. Show that $(y_n)$ converges and find the limit.

\newcommand{\fanin}{\forall \nin}
\pf The book tells us that $y_n \leq 1 + 2\sqrt{p} ~\fanin$, so $(y_n)$ is bounded above. Since $(y_n)$ is increasing, we know that $\lim(y_n) = \sup\{y_n : \nin\}$, and we call this value $y*$. Since this sequence is recursively defined, $\lim(y_{n+1}) = y^*$. Thus,

\begin{gather*}
y^* = \sqrt{p + y^*}\\
(y^*)^2 = p + y^*\\
(y^*) - y^* = p\\
(y^*)^2 - y^* - p = 0\\
\end{gather*}

so by the quadratic formula, $y* \in \{\frac{1 + \sqrt{1 + 4p}}{2}, \frac{1 - \sqrt{1 + 4p}}{2}\}$. We can rule out the second element by noting that:

\begin{gather*}
p > 0\\
4p > 0\\
1 + 4p > 1\\
\sqrt{1 + 4p} > 1\\
-\sqrt{1 + 4p} < -1\\
1 - \sqrt{1 + 4p} < 0\\
\end{gather*}

and since $y_n > 0$, it is clear that $y^* = \frac{1 + \sqrt{1 + 4p}}{2}$.\done

\section*{Section 3.3, Ex. 9}

Let $A$ be an infinite subset of $\mathbb{R}$ that is bounded above and let $u := \sup A$. Show there exists an increasing sequence $(x_n)$ with $x_n \in A ~ \fanin$ such that $u = \lim(x_n)$.

\pf By the Supremum Principle, for all $\epsilon > 0$, $\exists x_\epsilon \in A$ such that $u \geq x_\epsilon > x - \epsilon$. Pick some decreasing sequence of $(\epsilon_n)$ that converges to zero, and then $(x_n) = ({x_\epsilon}_n)$. \done

\section*{Section 3.3, Ex. 11}

Let $x_n := 1/1^1 + 1/2^2 + \cdots + 1/n^2$ for each $\nin$. Prove that $(x_n)$ is increasing and bounded, and hence converges.

\pf We want to show that $(x_n)$ is increasing. First note that $(x_n)$ is a sum of positive terms, it is positive. Thus, since $x_{n+1} = x_n + 1/(n+1)^2$, it is clear that $(x_n)$ is increasing.\\

The book tells us that $1/k^2 \leq 1/k(k-1) = 1/(k-1) - 1/k$ when $k \geq 2$. Noting that $x_k - x_{k-1} = 1/k^2$ and also that $\lim(1/k^2) = 0$, we have that $\lim(x_k - x_{k-1}) = 0$. This tells us that for large enough $k$, we have $|x_k - x_{k-1}| < \epsilon$, so by the Cauchy Criterion $(x_n)$ converges. This infers that it is bounded above. \done

\section*{Section 3.4, Ex 2}

Use the method of Example 3.4.3(b) to show that if $0 < c < 1$, then $\lim(c^{1/n}) = 1$.

\pf We begin by showing that $(z_n)$ is increasing. Consider that

\begin{gather*}
c < 1\\
c^{1/n} < 1\\
c^{1/n} \cdot c < 1 \cdot c < 1\\
c^{1/n + 1} < c < 1\\
c^{(n + 1)/n} < c < 1\\
c^{1/n} < c^{1/(n+1)} < 1\\
\end{gather*}

which tells us that $(z_n)$ is both increasing and bounded above by 1. This confirms that $\lim(z_n)$ exists, so let us call it $z^*$. Thus, the subsequence $(z_{2n})$ exists and converges to $z^*$.\\

Since $z_{2n} = c^{1/2n} = \sqrt{c^{1/n}}$, we know $\lim(z_{2n}) = \sqrt{z^*}$. But since $\lim(z_{2n}) = z^*$,  we have that $z^* = \sqrt{z^*}$. Thus, $z^* \in {0,1}$. Since $z_1 = c > 0$, and $(z_n)$ is increasing, we know that $z^* = 1$. \done

\section*{Section 3.4, Ex 4}

\subsection*{Part A}

Show that $(x_n) = (1 - (-1)^n + 1/n)$ diverges.

\pf $\lim(x_{2n}) = \lim(1 - 1 + 1/2n) = 0$, but $\lim(x_{2n - 1}) = \lim(1 + 1 + \frac{1}{2n - 1}) = 2$, so $(x_n)$ contains two convergent subsequences whose limits are unequal. Thus, $(x_n)$ diverges. \done

\subsection*{Part B}

Show that $(y_n) = (\sin(\frac{n\pi}{4}))$ diverges.

\pf $y_{8n} = \sin(\frac{8n\pi}{4}) = \sin(2n\pi) = 0$, which is a constant sequence. $y_{8n-1} = \sin(\frac{(8n - 1)\pi}{4}) = \sin(2n\pi - \pi/4) = -1$, which is also a constant sequence. Thus, $(y_n)$ contains two convergent subsequences whose limits are unequal. Thus, $(y_n)$ diverges. \done

\section*{Section 3.4, Ex 5}

Let $X = (x_n)$ and $Y = (y_n)$ be given sequences, and let the ``shuffled'' sequence $Z = (z_n)$ be defined by $z_{2n - 1} := x_n$ and $z_{2n} := y_n$. Show that $Z$ is convergent if and only if both $X$ and $Y$ are convergent and $\lim X = \lim Y$.

\pf We prove the logical equivalence of these statements in each direction separately.\\

($\rightarrow$) Assume that $Z$ converges to $z^*$, then any subsequence of $Z$ also converges to $z^*$. Thus, $\lim X = \lim Y = z^*$.\\

($\leftarrow$) Assume $X$ and $Y$ converge and that $\lim X = \lim Y$, but that $\lim Z = z^* \neq \lim X$. Then $|z^* - \lim X | = c$, so choose $\epsilon < c$, then $\exists M(\epsilon) \nin$ such that

\begin{equation*}
|z_{2n} - \lim X | < \epsilon \text{~and~} |z_{2n-1} - \lim X| < \epsilon
\end{equation*}

That is, all $z_j$ for $j$ even or odd lie in an $\epsilon$-neighborhood of $c$, so that for any choice of $\delta > 0$, only finitely many $z_j$ lie in $V_\delta(z^*)$. \done

\section*{Section 3.4, Ex 7}

\subsection*{Part A}

Find $\lim((1 + 1/n^2)^{n^2})$ if it exists.

\pf The sequence $(1 + 1/n^2)^{n^2}$ is a subsequence of $(1 + 1/n)^n$. Since $(1 + 1/n)^n \rightarrow e$, we know that $(1 + 1/n^2)^{n^2} \rightarrow e$. \done

\subsection*{Part D}

Find $\lim((1 + 2/n)^{n})$ if it exists.

\pf Let $n = 2j$, then $(1 + 2/2j)^{2j} = (1 + 1/j)^{2j} = (1 + 1/j)^j \cdot (1 + 1/j)^j$. Since we know $\lim(1 + 1/j)^j = e$, then by the limit laws we know that $\lim(1 + 2/n)^{n} = e^2$. \done

\subsection*{Part C}

Find $\lim((1 + 1/n^2)^{2n^2})$ if it exists.

\pf Take $m = n^2$, so that is becomes clear that we are working with a subsequence of $(1 + 1/m)^{2m}$. In Part D, we showed that $(1 + 1/m)^{2m} = (1 + 1/m) \cdot (1 + 1/m)$ and that the limit of this sequence is $e^2$. Thus, the $\lim((1 + 1/n^2)^{2n^2}) = e^2$. \done

\subsection*{Part B}

Find $\lim(1 + 1/2n)^n$ if it exists.

\pf Take $m = 2n$, so that it becomes clear that $(1 + 1/2n)^{2n}$ is a subsequence of $(1 + 1/n)^n$ and thus $\lim(1 + 1/2n)^{2n} = e$. We know from the limit laws that the root of a convergent sequence converges to the root of the sequence's limit. Thus, $\lim(\sqrt{(1 + 1/2n)^{2n}}) = \lim(1 + 1/2n)^n = \sqrt{e}$. \done

\end{document}
