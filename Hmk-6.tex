%%%%%%%%%%%%%%%%%%%%%%%%%%%%%%%%%%%%%%%%%
% Short Sectioned Assignment
% LaTeX Template
% Version 1.0 (5/5/12)
%
% This template has been downloaded from:
% http://www.LaTeXTemplates.com
%
% Original author:
% Frits Wenneker (http://www.howtotex.com)
%
% License:
% CC BY-NC-SA 3.0 (http://creativecommons.org/licenses/by-nc-sa/3.0/)
%
%%%%%%%%%%%%%%%%%%%%%%%%%%%%%%%%%%%%%%%%%

%----------------------------------------------------------------------------------------
%	PACKAGES AND OTHER DOCUMENT CONFIGURATIONS
%----------------------------------------------------------------------------------------

\documentclass[paper=a4, fontsize=11pt]{scrartcl} % A4 paper and 11pt font size

\usepackage[T1]{fontenc} % Use 8-bit encoding that has 256 glyphs
\usepackage{fourier} % Use the Adobe Utopia font for the document - comment this line to return to the LaTeX default
\usepackage[english]{babel} % English language/hyphenation
\usepackage{amsmath,amsfonts,amsthm} % Math packages

\usepackage{lipsum} % Used for inserting dummy 'Lorem ipsum' text into the template

\usepackage{sectsty} % Allows customizing section commands
\allsectionsfont{\centering \normalfont\scshape} % Make all sections centered, the default font and small caps

\usepackage{fancyhdr} % Custom headers and footers
\pagestyle{fancyplain} % Makes all pages in the document conform to the custom headers and footers
\fancyhead{} % No page header - if you want one, create it in the same way as the footers below
\fancyfoot[L]{} % Empty left footer
\fancyfoot[C]{} % Empty center footer
\fancyfoot[R]{\thepage} % Page numbering for right footer
\renewcommand{\headrulewidth}{0pt} % Remove header underlines
\renewcommand{\footrulewidth}{0pt} % Remove footer underlines
\setlength{\headheight}{13.6pt} % Customize the height of the header

\numberwithin{equation}{section} % Number equations within sections (i.e. 1.1, 1.2, 2.1, 2.2 instead of 1, 2, 3, 4)
\numberwithin{figure}{section} % Number figures within sections (i.e. 1.1, 1.2, 2.1, 2.2 instead of 1, 2, 3, 4)
\numberwithin{table}{section} % Number tables within sections (i.e. 1.1, 1.2, 2.1, 2.2 instead of 1, 2, 3, 4)

\setlength\parindent{0pt} % Removes all indentation from paragraphs - comment this line for an assignment with lots of text

%----------------------------------------------------------------------------------------
%	TITLE SECTION
%----------------------------------------------------------------------------------------

\newcommand{\horrule}[1]{\rule{\linewidth}{#1}} % Create horizontal rule command with 1 argument of height

\title{	
\normalfont \normalsize 
\textsc{University of Tennessee \\ Department of Mathematics} \\ [25pt] % Your university, school and/or department name(s)
\horrule{0.5pt} \\[0.4cm] % Thin top horizontal rule
\huge MATH 447 - Homework 6 \\ % The assignment title
\horrule{2pt} \\[0.5cm] % Thick bottom horizontal rule
}

\author{Robert D. French} % Your name

\date{\normalsize\today} % Today's date or a custom date

\begin{document}

\maketitle % Print the title

%----------------------------------------------------------------------------------------
%	PROBLEM 1
%----------------------------------------------------------------------------------------

\section*{Section 3.3, Ex. 3}
\newcommand{\nin}{n \in \mathbb{N}}
\newcommand{\pf}{\paragraph{Proof.}}
\newcommand{\done}{$\blacksquare$}

Let $x_1 \geq 2$ and $x_{n+1} := 1 + \sqrt{x_n - 1}$ for $\nin$. Show that $(x_n)$ is decreasing and bounded below by $2$. Find the limit.

\pf We can demonstrate by induction that $(x_n)$ is bounded below by 2. First note that $x_1 \geq 2$ by by hypothesis. Thus,

\begin{gather*}
x_1 \geq 2 \\
x_1 - 1 \geq 1\\
\sqrt{x_1 - 1} \geq 1\\
1 + \sqrt{x_1 - 1} \geq 1 + 1\\
x_2 \geq 2
\end{gather*}

so we have a basis for induction. Assume now that $x_n \geq 2$, and we will use this to demonstrate that $x_{n+1} \geq 2$:

\begin{gather*}
x_n \geq 2 \\
x_n - 1 \geq 1\\
\sqrt{x_n - 1} \geq 1\\
1 + \sqrt{x_n - 1} \geq 1 + 1\\
x_{n+1} \geq 2
\end{gather*}

Thus we have confirmed that $(x_n)$ is bounded below by 2.\\

We can also demonstrate by induction that $(x_n)$ is decreasing. Begin by noting that $x_1 \geq 2$, which implies that $x_1 - 1 \geq 1$. This tells us that $\sqrt{x_1} \geq 1$. Taking the difference of the previous equations tells us that $(x_1 - 1) - (\sqrt{x_1 - 1}) \geq 0$, and thus that $x_1 \geq 1 + \sqrt{x_1 - 1} = x_2$. Since $x_1 \geq x_2$, we have established a basis for induction.\\

Now assume that $x_{n-1} \geq x_n$ and recall that since 2 is a lower bound, $x_n - 1 \geq 1$. We aim to show that this implies $x_n \geq x_{n+1}$.

\begin{gather*}
x_{n-1} \geq x_n\\
x_{n-1} - 1 \geq x_n - 1\\
\sqrt{x_{n-1} - 1} \geq \sqrt{x_n - 1}\\
1 + \sqrt{x_{n-1} - 1} \geq 1 + \sqrt{x_n - 1}\\
x_n \geq x_{n+1}
\end{gather*}

Thus we have shown that $(x_n)$ decreases.\\

Since $(x_n)$ is decreasing, bounded below, and recursively defined, we may calculate its limit by acknowledging that $\lim(x_n) = \lim(x_{n+1})$, and for convenience we shall denote this $x*$. Thus,

\begin{gather*}
x^* = 1 + \sqrt{x^* - 1}\\
x^* + 1 = \sqrt{x^* - 1}\\
(x^* + 1)^2 = x^* - 1\\
{x^*}^2 - 3x^* + 2 = 0
\end{gather*}

The solutions of the final quadratic equation are 1 and 2, however, since 2 is a lower bound for $(x_n)$, we know its limit may not be 1. Thus, $\lim(x_n) = 2$.

\end{document}
