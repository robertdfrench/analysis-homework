%%%%%%%%%%%%%%%%%%%%%%%%%%%%%%%%%%%%%%%%%
% Short Sectioned Assignment
% LaTeX Template
% Version 1.0 (5/5/12)
%
% This template has been downloaded from:
% http://www.LaTeXTemplates.com
%
% Original author:
% Frits Wenneker (http://www.howtotex.com)
%
% License:
% CC BY-NC-SA 3.0 (http://creativecommons.org/licenses/by-nc-sa/3.0/)
%
%%%%%%%%%%%%%%%%%%%%%%%%%%%%%%%%%%%%%%%%%

%----------------------------------------------------------------------------------------
%	PACKAGES AND OTHER DOCUMENT CONFIGURATIONS
%----------------------------------------------------------------------------------------

\documentclass[paper=a4, fontsize=11pt]{scrartcl} % A4 paper and 11pt font size

\usepackage[T1]{fontenc} % Use 8-bit encoding that has 256 glyphs
\usepackage{fourier} % Use the Adobe Utopia font for the document - comment this line to return to the LaTeX default
\usepackage[english]{babel} % English language/hyphenation
\usepackage{amsmath,amsfonts,amsthm} % Math packages

\usepackage{lipsum} % Used for inserting dummy 'Lorem ipsum' text into the template

\usepackage{sectsty} % Allows customizing section commands
\allsectionsfont{\centering \normalfont\scshape} % Make all sections centered, the default font and small caps

\usepackage{fancyhdr} % Custom headers and footers
\pagestyle{fancyplain} % Makes all pages in the document conform to the custom headers and footers
\fancyhead{} % No page header - if you want one, create it in the same way as the footers below
\fancyfoot[L]{} % Empty left footer
\fancyfoot[C]{} % Empty center footer
\fancyfoot[R]{\thepage} % Page numbering for right footer
\renewcommand{\headrulewidth}{0pt} % Remove header underlines
\renewcommand{\footrulewidth}{0pt} % Remove footer underlines
\setlength{\headheight}{13.6pt} % Customize the height of the header

\numberwithin{equation}{section} % Number equations within sections (i.e. 1.1, 1.2, 2.1, 2.2 instead of 1, 2, 3, 4)
\numberwithin{figure}{section} % Number figures within sections (i.e. 1.1, 1.2, 2.1, 2.2 instead of 1, 2, 3, 4)
\numberwithin{table}{section} % Number tables within sections (i.e. 1.1, 1.2, 2.1, 2.2 instead of 1, 2, 3, 4)

\setlength\parindent{0pt} % Removes all indentation from paragraphs - comment this line for an assignment with lots of text

%----------------------------------------------------------------------------------------
%	TITLE SECTION
%----------------------------------------------------------------------------------------

\newcommand{\horrule}[1]{\rule{\linewidth}{#1}} % Create horizontal rule command with 1 argument of height

\title{	
\normalfont \normalsize 
\textsc{University of Tennessee \\ Department of Mathematics} \\ [25pt] % Your university, school and/or department name(s)
\horrule{0.5pt} \\[0.4cm] % Thin top horizontal rule
\huge MATH 447 - Homework 5 \\ % The assignment title
\horrule{2pt} \\[0.5cm] % Thick bottom horizontal rule
}

\author{Robert D. French} % Your name

\date{\normalsize\today} % Today's date or a custom date

\begin{document}

\maketitle % Print the title

%----------------------------------------------------------------------------------------
%	PROBLEM 1
%----------------------------------------------------------------------------------------

\section*{Section 3.1}

\paragraph{Problem 10.} Prove that if $\lim(x_n) = 0$ and if $x > 0$, then there exists a natural number $M$ such that $x_n > 0$ for all $n \geq M$.

\paragraph{Proof.} By the definition of limit, for each positive $\epsilon$ such an $M(\epsilon)$ can be found. Thus, take $\epsilon$ to be smaller than $x$, so that $x_n \in V_\epsilon (x) = (x - \epsilon, x + \epsilon)$ whenever $n \geq M$. Since $\epsilon < x$, $x - \epsilon \in \mathbb{P}$ so $x_n > 0$. $\blacksquare$

\paragraph{Problem 14.} Let $b \in \mathbb{R}$ satisfy $0 < b < 1$. Show that $\lim(nb^n) = 0$.

\paragraph{Proof.} We first begin by proving the following Lemma: If $(k_n)^2 \rightarrow 0$ and $k_n < 0$, then $k_n \rightarrow 0$.\\

By hypothesis, given an $\epsilon > 0$, there exists some $M(\epsilon) \in \mathbb{N}$ such that  $|(k_n)^2 - 0| < \epsilon$ whenever $n \geq M(\epsilon)$. Further, $|(k_n)^2 - 0|$ = $\left| \left| k_n \right| \cdot \left| k_n \right| \right|$ = $|k_n|^2$ so that we have $|k_n| < \sqrt{\epsilon}$. Since the absolute value of $k_n$ is smaller than any arbitrary positive number, and we know that $k_n$ < 0 for all choices of $n$, we have that $k_n \rightarrow 0$.\\

Now we may proceed with the main portion of our argument. Since $b \in (0,1)$, we may define a sequence $(k_n)$ such that $b^{1/n} = 1 + k_n$ for some $k_n < 0$. This implies that $b = (1 + k_n)^n$.\\

The Binomial Theorem tells us that $b = 1 + n k_n + \frac{1}{2} n (n-1) (k_n)^2 + \cdots \geq 1 + \frac{1}{2} n (n-1) (k_n)^2$, so we have that $b - 1 \geq \frac{1}{2} n (n-1) (k_n)^2$. Thus, $\frac{2b - 2}{n(n-1)} \geq (k_n)^2$.\\

We know from previous work that $\frac{1}{n(n-1)} \rightarrow 0$, and since $\frac{2b - 2}{n(n-1)}$ is a constant times a convergent expression, we know that its limit is zero as well. Since $(k_n)^2$ is bounded above by a sequence which converges to zero, and below by the constant sequence $(0)$, we have that $(k_n)^2 \rightarrow 0$. Thus, by our earlier Lemma, $k_n$	

\end{document}
