%%%%%%%%%%%%%%%%%%%%%%%%%%%%%%%%%%%%%%%%%
% Short Sectioned Assignment
% LaTeX Template
% Version 1.0 (5/5/12)
%
% This template has been downloaded from:
% http://www.LaTeXTemplates.com
%
% Original author:
% Frits Wenneker (http://www.howtotex.com)
%
% License:
% CC BY-NC-SA 3.0 (http://creativecommons.org/licenses/by-nc-sa/3.0/)
%
%%%%%%%%%%%%%%%%%%%%%%%%%%%%%%%%%%%%%%%%%

%----------------------------------------------------------------------------------------
%	PACKAGES AND OTHER DOCUMENT CONFIGURATIONS
%----------------------------------------------------------------------------------------

\documentclass[paper=a4, fontsize=11pt]{scrartcl} % A4 paper and 11pt font size

\usepackage[T1]{fontenc} % Use 8-bit encoding that has 256 glyphs
\usepackage{fourier} % Use the Adobe Utopia font for the document - comment this line to return to the LaTeX default
\usepackage[english]{babel} % English language/hyphenation
\usepackage{amsmath,amsfonts,amsthm} % Math packages

\usepackage{lipsum} % Used for inserting dummy 'Lorem ipsum' text into the template

\usepackage{sectsty} % Allows customizing section commands
\allsectionsfont{\centering \normalfont\scshape} % Make all sections centered, the default font and small caps

\usepackage{fancyhdr} % Custom headers and footers
\pagestyle{fancyplain} % Makes all pages in the document conform to the custom headers and footers
\fancyhead{} % No page header - if you want one, create it in the same way as the footers below
\fancyfoot[L]{} % Empty left footer
\fancyfoot[C]{} % Empty center footer
\fancyfoot[R]{\thepage} % Page numbering for right footer
\renewcommand{\headrulewidth}{0pt} % Remove header underlines
\renewcommand{\footrulewidth}{0pt} % Remove footer underlines
\setlength{\headheight}{13.6pt} % Customize the height of the header

\numberwithin{equation}{section} % Number equations within sections (i.e. 1.1, 1.2, 2.1, 2.2 instead of 1, 2, 3, 4)
\numberwithin{figure}{section} % Number figures within sections (i.e. 1.1, 1.2, 2.1, 2.2 instead of 1, 2, 3, 4)
\numberwithin{table}{section} % Number tables within sections (i.e. 1.1, 1.2, 2.1, 2.2 instead of 1, 2, 3, 4)

\setlength\parindent{0pt} % Removes all indentation from paragraphs - comment this line for an assignment with lots of text

%----------------------------------------------------------------------------------------
%	TITLE SECTION
%----------------------------------------------------------------------------------------

\newcommand{\horrule}[1]{\rule{\linewidth}{#1}} % Create horizontal rule command with 1 argument of height

\title{	
\normalfont \normalsize 
\textsc{University of Tennessee \\ Department of Mathematics} \\ [25pt] % Your university, school and/or department name(s)
\horrule{0.5pt} \\[0.4cm] % Thin top horizontal rule
\huge MATH 447 - Homework 5 \\ % The assignment title
\horrule{2pt} \\[0.5cm] % Thick bottom horizontal rule
}

\author{Robert D. French} % Your name

\date{\normalsize\today} % Today's date or a custom date

\begin{document}

\maketitle % Print the title


\section*{Section 3.1}

\newcommand{\prob}[1]{\subsection*{Problem {#1}}}
\newcommand{\subprob}[1]{\subsubsection*{{#1}}}
\newcommand{\pf}{\paragraph{Proof.}}
\newcommand{\done}{$\blacksquare$}

%%%%% Problem 3.1.10
\prob{10} Prove that if $\lim(x_n) = 0$ and if $x > 0$, then there exists a natural number $M$ such that $x_n > 0$ for all $n \geq M$.

\pf By the definition of limit, for each positive $\epsilon$ such an $M(\epsilon)$ can be found. Thus, take $\epsilon$ to be smaller than $x$, so that $x_n \in V_\epsilon (x) = (x - \epsilon, x + \epsilon)$ whenever $n \geq M$. Since $\epsilon < x$, $x - \epsilon \in \mathbb{P}$ so $x_n > 0$.\done

\newcommand{\nin}{n \in \mathbb{N}}
\newcommand{\fanin}{\forall n \in \mathbb{N}}
%%%%% Problem 3.1.14
\prob{14} Let $b \in \mathbb{R}$ satisfy $0 < b < 1$. Show that $\lim(nb^n) = 0$.

\pf We know that $n^{1/n} > 1$ $\fanin$. This implies that $n^{1/n} = 1 + k_n$ for some $k_n > 0$, so that we have $bn^{1/n} = b + b k_n$. Thus, $nb^n = (b + b k_n)^n$.\\

By the Binomial Theorem we know that 

\begin{equation*}
nb^n = (b + b k_n)^n = b^n + n b^{n-1} b k_n + \frac{1}{2} n (n - 1) b^{n - 2} b^2 {k_n}^2 + \cdots
\end{equation*}

which tells us that $nb^n \geq b^n + \frac{1}{2} n (n-1) b^n {k_n}^2$. Factoring out $b^n$ and subtracting 1 yields

\begin{equation*}
(n - 1) \geq \frac{1}{2} n (n - 1) {k_n}^2
\end{equation*}

so that we have $\frac{2}{n} \geq {k_n}^2$.

%%%%% Problem 3.1.17
\prob{17}
\subprob{Hint} If $n \geq 3$, then $0 < 2^n/n! \leq 2(\frac{2}{3})^{n-2}$.

\pf Let $n = 3$, then $\frac{2^3}{3!} = \frac{8}{6} \leq 2(\frac{2}{3}) = \frac{4}{3} = \frac{8}{6}$. Thus, we have established a basis for induction.\\

Now assume that $0 < \frac{2^{n-1}}{(n-1)!} \leq 2\left(\frac{2}{3}\right)^{n-3}$. Multiplying by $2$ yields $ 0 < \frac{2^n}{(n-1)!} \leq 2\left(\frac{2^{n-2}}{3^{n-3}}\right)$. Since $n \geq 3$, we have $0 < \frac{1}{n} \leq \frac{1}{3}$ so that

\begin{equation*}
0 < \frac{2^n}{(n-1)!} \cdot \frac{1}{n} = \frac{2^n}{n!} \leq 2 \cdot \left(\frac{2^{n-2}}{3^{n-3}}\right) \cdot \frac{1}{3} = 2 \left( \frac{2}{3} \right)^{n-2} 
\end{equation*}

whence $0 < \frac{2^n}{n!} \leq 2\left(\frac{2}{3}\right)^{n-2}$. Therefore the hypothesis holds by the Principal of Mathematical Inducation. \done

\subprob{Main Problem} Show that $\lim(2^n/n!) = 0$.

\pf We know that $0 < \frac{2^n}{n!} \leq 2\left(\frac{2}{3}\right)^{n-2}$ when $n \geq 3$, so if we can show that $2\left(\frac{2}{3}\right)^{n-2} \rightarrow 0$, then $\frac{2^n}{n!} \rightarrow 0$ by the Squeeze Theorem.\\

Take $b = 2/3$ and $n = m - 2$, then by Exercise 3.1.14 we know $nb^n \rightarrow 0$. Since $2 \leq m - 2$ $\forall m > 3$, we have that $0 < 2(\frac{2}{3})^{m-2} \leq (m-2)b^{m-2}$ so that $2\left(\frac{2}{3}\right)^{m-2} \rightarrow 0$. Thus, by the Squeeze Theorem, $2^n / n! \rightarrow 0$. \done

%%%%% Problem 3.1.18
\prob{18} If $\lim(x_n) = x > 0$, show that there exists a natural number $K$ such that if $n \geq K$, then $\frac{1}{2}x < x_n < 2x$.

\pf Pick $\epsilon < \frac{1}{2} x$, then by the definition of limit $\exists K$ such that $x_n \in V_\epsilon(x) \forall n \geq K$. Since $V_\epsilon(x) = (x - x/2, x + x/2) = (x/2, 3x/2)$, we know $V_\epsilon(x) \subseteq (x/2, 2x)$, demonstrating that such a $K$ exists. \done

\section*{Section 3.2}
%%%%% Problem 3.2.7
\prob{7} If $(b_n)$ is a bounded sequence and $\lim(a_n) = 0$, show that $\lim(a_n b_n) = 0$. Explain why Thm 3.2.3 cannot be used.

\paragraph{Remark.} We may not apply Theorem 3.2.3 directly because it requires that $(b_n)$ is convergent, but we know only that $(b_n)$ is bounded. However, we may employ Theorem 3.2.3 for certain features of our argument.

\pf Since $(b_n)$ is bounded, we have

\begin{equation*}
I_b = \inf(b_n) \leq b_j \leq \sup(b_n) = S_b ~ \forall j \in \mathbb{N}
\end{equation*}

which implies

\begin{equation*}
I_b |a_j|  = \inf(b_n) |a_j| \leq b_j \cdot |a_j| \leq \sup(b_n) \cdot |a_j| = S_b \cdot |a_j| ~ \forall j \in \mathbb{N}
\end{equation*}

Thus if $\inf(b_n)|a_j| \rightarrow 0$ and $\sup(b_n) |a_j| \rightarrow 0$, we will be confident that $b_j |a_j| \rightarrow 0$.\\

Since $\inf(b_n)$ and $\sup(b_n)$ are constants, and since we know $|a_j| \rightarrow 0$ due to the fact that $a_j \rightarrow 0$, we have that $\inf(b_n) |a_j| \rightarrow 0$ and $\sup(b_n) |a_j| \rightarrow 0$ by Theorem 3.2.3. Thus, $b_j \cdot |a_j| \rightarrow 0$.\\

Therefore, given $\epsilon > 0$, we know $\exists K(\epsilon)$ such that $\left| b_j \cdot |a_j| - 0 \right| < \epsilon ~ \forall j \geq K(\epsilon)$.\\

What we want to show is that $|b_j \cdot a_j - 0| < \epsilon ~ \forall j \geq M(\epsilon)$. By the above work, we know $|b_j \cdot |a_j| - 0| < \epsilon$ which tells us that $|b_j \cdot |a_j - 0| < \epsilon$ and thereforce $\lim(a_n b_n) = 0$. \done

%%%%% Problem 3.2.9
\prob{9} Let $y_n := \sqrt{n+1} - \sqrt{n}$ for $n \in \mathbb{N}$. Show that $(\sqrt{n} y_n)$ converges. Find the limit.

\pf We want to show that there exists (and we can find) some $L$ such that given $\epsilon > 0$, $|\sqrt{n+1} - \sqrt{n} - L| < \epsilon$. Using the Ratio Test,

\begin{equation*}
\frac{y_{n+1}}{y_n} = \frac{\sqrt{n+2} - \sqrt{n+1}}{\sqrt{n+1} - \sqrt{n}}
\end{equation*}

we consider the conjugate of $y_{n+1}$

\begin{equation*}
\frac{y_{n+1}}{y_n} = \frac{\sqrt{n+2} - \sqrt{n+1}}{\sqrt{n+1} - \sqrt{n}} \left( \frac{\sqrt{n+2} + \sqrt{n+1}}{\sqrt{n+2} + \sqrt{n+1}} \right)
\end{equation*}

In this case, the numerator simplifies to $1$, while the denominator resolves to the following expression which we call $\phi_n$:

\begin{equation*}
\phi_n = \sqrt{n+1}\left(\sqrt{n+2} + \sqrt{n+1} - 1\right) - \sqrt{n}(\sqrt{n+2})
\end{equation*}

Our immediate goal is to show that $\frac{y_{n+1}}{y_n} = 1/\phi_n$ converges. Well, we know $\sqrt{n+1} > \sqrt{n}$, and since $\sqrt{n+1} - 1$ is positive, we know that $\sqrt{n+2} + \sqrt{n+1} - 1 > \sqrt{n+2}$. Multiplying these two facts gives us

\begin{equation*}
\sqrt{n+1}(\sqrt{n+2} + \sqrt{n+1} - 1) > \sqrt{n}\sqrt{n+2}
\end{equation*}

By definition, subtracting the right hand side of the above inequality from the left yields a positive number, and since this difference is exactly $\phi_n$, we know that $\phi_n > 0$.\\

By the Archimedean Property, we know that for each $\nin$ there is a least $M > \phi_n$ so that $K = M - 1 \leq \phi_n$. Thus, $1/K \geq 1/\phi_n$. Since $\phi_n > 0$, we know $1/n \geq 1/\phi_n > 0$ when $n \geq K$ so that $1/\phi_n \rightarrow 0$. Since $0<1$, we know by the ratio test that $y_n$ converges and $\lim(y_n) = 0$.\done

%%%%% Problem 3.2.11
\prob{11}
\subprob{Part A} Find $\lim\left((3\sqrt{n})^{1/2n}\right)$
\pf We know that $\sqrt[n]{3\sqrt{n}} > 0$ since the domain function $3 \sqrt{n} > 0$. So if $\sqrt[n]{3\sqrt{n}}$ can be shown to converge to $L_1$, then $\sqrt{\sqrt[n]{3\sqrt{n}}} \rightarrow L_0 = \sqrt{L_1}$.

\begin{equation*}
\sqrt[n]{3 \sqrt{n}} = (3 \sqrt{n})^{1/n} = 3^{1/n} \sqrt{n}^{1/n} = 3^{1/n} n^{1/2n} = 3^{1/n} \sqrt{n^{1/n}}
\end{equation*}

If $3^{1/n}$ and $\sqrt{n^{1/n}}$ can be shown to converge to $L_3$ and $L_2$ respectively, then $L_1 = L_3 \cdot L_2$, so this is our goal.\\

Since $n^{1/n} > 0$ and we have shown by earlier work that $n^{1/n} \rightarrow 1$, then by Theorem 3.2.10 we have that $\sqrt{n^{1/n}} \rightarrow \sqrt{1}$ or rather $L_2 = 1$.\\

Since $1 < 3 < n$ whenever $n \geq 3$, we know that $1^{1/n} < 3^{1/n} < n^{1/n}$ whenever $n \geq 3$. Thus, since $n^{1/n} \rightarrow 1$, we have that $3^{1/n} \rightarrow 1$, so $L_3 = 1$.\\

Thus, $L_1 = L_3 L_2 = 1 \cdot 1 = 1$, so that $L_0 = \sqrt{L_1} = \sqrt{1} = 1$. Thus, $\lim\left((3\sqrt{n})^{1/2n}\right) = 1$. \done

\subprob{Part B} Find $(\sqrt{n^2 + 5n} - n)$
\pf

%%%%% Problem 3.2.12
\newcommand{\anp}{a^{n+1}}
\newcommand{\bnp}{b^{n+1}}
\prob{12} If $0 < a < b$, determine $\left(\frac{\anp + \bnp}{a^n + b^n}\right)$.
\pf First note that

\begin{equation*}
\left(\frac{\anp + \bnp}{a^n + b^n}\right) = 
\left(\frac{\anp}{a^n + b^n} + \frac{\bnp}{a^n + b^n}\right) = 
a \left(\frac{a^n}{a^n + b^n}\right) + b \left(\frac{b^n}{a^n + b^n}\right) 
\end{equation*}

and since $a^n + b^n = a^n(1 + (b/a)^n) = b^n((a/b)^n + 1)$ we can write 

\begin{equation*}
\left(\frac{\anp + \bnp}{a^n + b^n}\right) = 
a \left(\frac{1}{1 + (b/a)^n}\right) + b \left(\frac{1}{(a/b)^n + 1}\right) 
\end{equation*}

which encourages us to focus our discussion on $(b/a)^n$ and $(a/b)^n$.\\

Since $b > a > 0$, we know that $b/a > 1$ so that $(b/a)^n + 1$ is unbounded. The Archimedean Property guarantees us an $M$ bigger than $(b/a)^n + 1$, so that $1/M > \frac{1}{(b/a)^n + 1}$ for all sufficiently large $n$. Thus $\frac{1}{(b/a)^n + 1} \rightarrow 0$ and so does $\frac{a}{(b/a)^n + 1}$ since it is a constant multiple.\\

Knowing that $b > a > 0$ also tells us that $a/b < 1$ so that $a/b \in (0,1)$. We know from earlier work that $\lim(nc^n) = 0$ when $c \in (0,1)$, and since $(a/b)^n < n(a/b)^n$ for all $n$, we have that $\lim((a/b)^n) = 0$ as well. Thus, $\lim(\frac{b}{(a/b)^n + 1}) = b$.\\

Since we have expressed our original sequence in terms of the sum of two convergent sequences, we have that 

\begin{equation*}
\lim\left(\frac{\anp + \bnp}{a^n + b^n}\right) = b ~ \blacksquare
\end{equation*}

%%%%% Problem 3.2.13
\prob{13} If $a > 0$, $b > 0$, show that $\lim\left(\sqrt{(n+a)(n+b)} - n\right) = (a + b)/2$.
\pf Consider multiplying by a conjugate:

\begin{gather*}
\sqrt{(n+a)(n+b)} - n \\
(\sqrt{(n+a)(n+b)} - n)\left(\frac{\sqrt{(n+a)(n+b)} + n}{\sqrt{(n+a)(n+b)} + n}\right)\\ 
\frac{(n+a)(n+b) - n^2}{\sqrt{(n+a)(n+b)} + n}\\ 
\frac{n(a+b) + ab}{\sqrt{(n+a)(n+b)} + n}\\
\frac{n(a+b)}{\sqrt{(n+a)(n+b)} + n} + \frac{ab}{\sqrt{(n+a)(n+b)} + n}
\end{gather*}

We note that the second term $(ab)/(\sqrt{(n+a)(n+b)} + n)$ has a fixed numerator and an unbounded denominator, so the Archimedean Property affirms that the quotient will tend towards zero. Thus we break apart the first term into

\begin{equation*}
\frac{n(a+b)}{\sqrt{(n+a)(n+b)} + n} = \frac{na}{\sqrt{(n+a)(n+b)} + n} + \frac{nb}{\sqrt{(n+a)(n+b)} + n}
\end{equation*}

We note that since $na$, $nb$, and $ab$ are all positive, we have

\begin{equation*}
n^2 + na + nb + ab > n^2
\end{equation*}

so we know that $\sqrt{(n+a)(n+b)} > n$. Thus,

\begin{equation*}
\frac{na}{\sqrt{(n+a)(n+b)} + n} + \frac{nb}{\sqrt{(n+a)(n+b)} + n} \leq \frac{na}{2n} + \frac{nb}{2n} = \frac{a + b}{2}
\end{equation*}

%%%%% Problem 3.2.14
\prob{14}
\subprob{Part A} Use the Squeeze Theorem to find $\lim(n^{1/n^2})$.
\pf We know from earlier work that $\lim(n^{1/n}) = 1$. We can infer from this that $n > n^{1/n}$ for sufficiently large values of $n$. This implies that $n^{1/n} > n^{1/n^2}$.\\

Also, consider that $n > 1^{n^2}$. This lets us know that $n^{1/n^2} > 1$. Therefore $1 \leq n^{1/n^2} \leq n^{1/n}$ for all sufficiently large $n$, and since $n^{1/n} \rightarrow 1$, we may employ the Squeeze Theorem to assert that $n^{1/n^2} \rightarrow 1$. \done

\subprob{Part B} Use the Squeeze Theorem to find $\lim((n!)^{1/n^2})$.
\pf

%%%%% Problem 3.2.15
\prob{15} Show that if $z_n := (a^n + b^n)^{1/n}$ where $0 < a < b$, then $\lim(z_n) = b$.
\pf

%%%%% Problem 3.2.23
\prob{23} Show that if $(x_n)$ and $(y_n)$ are convergent sequences, then the sequences $(u_n)$ and $(v_n)$ defined by $u_n := \max\{x_n, y_n\}$ and $v_n := \min\{x_n, y_n\}$ are also convergent.
\pf

%%%%% Problem 3.2.24
\prob{24} Show that if $(x_n)$, $(y_n)$, $(z_n)$ are convergent sequences, then the sequence $(w_n)$ defined by $w_n := \text{mid}\{x_n, y_n, z_n\}$ is also convergent.
\pf
\end{document}
