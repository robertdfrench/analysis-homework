%%%%%%%%%%%%%%%%%%%%%%%%%%%%%%%%%%%%%%%%%
% Short Sectioned Assignment
% LaTeX Template
% Version 1.0 (5/5/12)
%
% This template has been downloaded from:
% http://www.LaTeXTemplates.com
%
% Original author:
% Frits Wenneker (http://www.howtotex.com)
%
% License:
% CC BY-NC-SA 3.0 (http://creativecommons.org/licenses/by-nc-sa/3.0/)
%
%%%%%%%%%%%%%%%%%%%%%%%%%%%%%%%%%%%%%%%%%

%----------------------------------------------------------------------------------------
%	PACKAGES AND OTHER DOCUMENT CONFIGURATIONS
%----------------------------------------------------------------------------------------

\documentclass[paper=a4, fontsize=11pt]{scrartcl} % A4 paper and 11pt font size

\usepackage[T1]{fontenc} % Use 8-bit encoding that has 256 glyphs
\usepackage{fourier} % Use the Adobe Utopia font for the document - comment this line to return to the LaTeX default
\usepackage[english]{babel} % English language/hyphenation
\usepackage{amsmath,amsfonts,amsthm} % Math packages

\usepackage{lipsum} % Used for inserting dummy 'Lorem ipsum' text into the template

\usepackage{sectsty} % Allows customizing section commands
\allsectionsfont{\centering \normalfont\scshape} % Make all sections centered, the default font and small caps

\usepackage{fancyhdr} % Custom headers and footers
\pagestyle{fancyplain} % Makes all pages in the document conform to the custom headers and footers
\fancyhead{} % No page header - if you want one, create it in the same way as the footers below
\fancyfoot[L]{} % Empty left footer
\fancyfoot[C]{} % Empty center footer
\fancyfoot[R]{\thepage} % Page numbering for right footer
\renewcommand{\headrulewidth}{0pt} % Remove header underlines
\renewcommand{\footrulewidth}{0pt} % Remove footer underlines
\setlength{\headheight}{13.6pt} % Customize the height of the header

\numberwithin{equation}{section} % Number equations within sections (i.e. 1.1, 1.2, 2.1, 2.2 instead of 1, 2, 3, 4)
\numberwithin{figure}{section} % Number figures within sections (i.e. 1.1, 1.2, 2.1, 2.2 instead of 1, 2, 3, 4)
\numberwithin{table}{section} % Number tables within sections (i.e. 1.1, 1.2, 2.1, 2.2 instead of 1, 2, 3, 4)

\setlength\parindent{0pt} % Removes all indentation from paragraphs - comment this line for an assignment with lots of text

%----------------------------------------------------------------------------------------
%	TITLE SECTION
%----------------------------------------------------------------------------------------

\newcommand{\horrule}[1]{\rule{\linewidth}{#1}} % Create horizontal rule command with 1 argument of height

\title{	
\normalfont \normalsize 
\textsc{University of Tennessee \\ Department of Mathematics} \\ [25pt] % Your university, school and/or department name(s)
\horrule{0.5pt} \\[0.4cm] % Thin top horizontal rule
\huge MATH 447 - Homework 5 \\ % The assignment title
\horrule{2pt} \\[0.5cm] % Thick bottom horizontal rule
}

\author{Robert D. French} % Your name

\date{\normalsize\today} % Today's date or a custom date

\begin{document}

\maketitle % Print the title


\section*{Section 3.1}

\newcommand{\prob}[1]{\subsection*{Problem {#1}}}
\newcommand{\subprob}[1]{\subsubsection*{{#1}}}
\newcommand{\pf}{\paragraph{Proof.}}
\newcommand{\done}{$\blacksquare$}

%%%%% Problem 3.1.10
\prob{10} Prove that if $\lim(x_n) = 0$ and if $x > 0$, then there exists a natural number $M$ such that $x_n > 0$ for all $n \geq M$.

\pf By the definition of limit, for each positive $\epsilon$ such an $M(\epsilon)$ can be found. Thus, take $\epsilon$ to be smaller than $x$, so that $x_n \in V_\epsilon (x) = (x - \epsilon, x + \epsilon)$ whenever $n \geq M$. Since $\epsilon < x$, $x - \epsilon \in \mathbb{P}$ so $x_n > 0$.\done

\newcommand{\nin}{n \in \mathbb{N}}
\newcommand{\fanin}{\forall n \in \mathbb{N}}
%%%%% Problem 3.1.14
\prob{14} Let $b \in \mathbb{R}$ satisfy $0 < b < 1$. Show that $\lim(nb^n) = 0$.

\pf We know that $n^{1/n} > 1$ $\fanin$. This implies that $n^{1/n} = 1 + k_n$ for some $k_n > 0$, so that we have $bn^{1/n} = b + b k_n$. Thus, $nb^n = (b + b k_n)^n$.\\

By the Binomial Theorem we know that 

\begin{equation*}
nb^n = (b + b k_n)^n = b^n + n b^{n-1} b k_n + \frac{1}{2} n (n - 1) b^{n - 2} b^2 {k_n}^2 + \cdots
\end{equation*}

which tells us that $nb^n \geq b^n + \frac{1}{2} n (n-1) b^n {k_n}^2$. Factoring out $b^n$ and subtracting 1 yields

\begin{equation*}
(n - 1) \geq \frac{1}{2} n (n - 1) {k_n}^2
\end{equation*}

so that we have $\frac{2}{n} \geq {k_n}^2$.

%%%%% Problem 3.1.17
\prob{17}
\subprob{Hint} If $n \geq 3$, then $0 < 2^n/n! \leq 2(\frac{2}{3})^{n-2}$.

\pf Let $n = 3$, then $\frac{2^3}{3!} = \frac{8}{6} \leq 2(\frac{2}{3}) = \frac{4}{3} = \frac{8}{6}$. Thus, we have established a basis for induction.\\

Now assume that $0 < \frac{2^{n-1}}{(n-1)!} \leq 2\left(\frac{2}{3}\right)^{n-3}$. Multiplying by $2$ yields $ 0 < \frac{2^n}{(n-1)!} \leq 2\left(\frac{2^{n-2}}{3^{n-3}}\right)$. Since $n \geq 3$, we have $0 < \frac{1}{n} \leq \frac{1}{3}$ so that

\begin{equation*}
0 < \frac{2^n}{(n-1)!} \cdot \frac{1}{n} = \frac{2^n}{n!} \leq 2 \cdot \left(\frac{2^{n-2}}{3^{n-3}}\right) \cdot \frac{1}{3} = 2 \left( \frac{2}{3} \right)^{n-2} 
\end{equation*}

whence $0 < \frac{2^n}{n!} \leq 2\left(\frac{2}{3}\right)^{n-2}$. Therefore the hypothesis holds by the Principal of Mathematical Inducation. \done

\subprob{Main Problem} Show that $\lim(2^n/n!) = 0$.

\pf We know that $0 < \frac{2^n}{n!} \leq 2\left(\frac{2}{3}\right)^{n-2}$ when $n \geq 3$, so if we can show that $2\left(\frac{2}{3}\right)^{n-2} \rightarrow 0$, then $\frac{2^n}{n!} \rightarrow 0$ by the Squeeze Theorem.\\

Take $b = 2/3$ and $n = m - 2$, then by Exercise 3.1.14 we know $nb^n \rightarrow 0$. Since $2 \leq m - 2$ $\forall m > 3$, we have that $0 < 2(\frac{2}{3})^{m-2} \leq (m-2)b^{m-2}$ so that $2\left(\frac{2}{3}\right)^{m-2} \rightarrow 0$. Thus, by the Squeeze Theorem, $2^n / n! \rightarrow 0$. \done

%%%%% Problem 3.1.18
\prob{18} If $\lim(x_n) = x > 0$, show that there exists a natural number $K$ such that if $n \geq K$, then $\frac{1}{2}x < x_n < 2x$.
\pf

\section*{Section 3.2}
%%%%% Problem 3.2.7
\prob{7} If $(b_n)$ is a bounded sequence and $\lim(a_n) = 0$, show that $\lim(a_n b_n) = 0$. Explain why Thm 3.2.3 cannot be used.
\pf

%%%%% Problem 3.2.9
\prob{9} Let $y_n := \sqrt{n+1} - \sqrt{n}$ for $n \in \mathbb{N}$. Show that $(\sqrt{n} y_n)$ converges. Find the limit.
\pf

%%%%% Problem 3.2.11
\prob{11}
\subprob{Part A} Find $\lim\left((3\sqrt{n})^{1/2n}\right)$
\pf
\subprob{Part B} Find $(\sqrt{n^2 + 5n} - n)$
\pf

%%%%% Problem 3.2.12
\newcommand{\anp}{a^{n+1}}
\newcommand{\bnp}{b^{n+1}}
\prob{12} If $0 < a < b$, determine $\left(\frac{\anp + \bnp}{a^n + b^n}\right)$.
\pf

%%%%% Problem 3.2.13
\prob{13} If $a > 0$, $b > 0$, show that $\lim\left(\sqrt{(n+a)(n+b)} - n\right) = (a + b)/2$.
\pf

%%%%% Problem 3.2.14
\prob{14}
\subprob{Part A} Use the Squeeze Theorem to find $\lim(n^{1/n^2})$.
\pf
\subprob{Part B} Use the Squeeze Theorem to find $\lim((n!)^{1/n^2})$.
\pf

%%%%% Problem 3.2.15
\prob{15} Show that if $z_n := (a^n + b^n)^{1/n}$ where $0 < a < b$, then $\lim(z_n) = b$.
\pf

%%%%% Problem 3.2.23
\prob{23} Show that if $(x_n)$ and $(y_n)$ are convergent sequences, then the sequences $(u_n)$ and $(v_n)$ defined by $u_n := \max\{x_n, y_n\}$ and $v_n := \min\{x_n, y_n\}$ are also convergent.
\pf

%%%%% Problem 3.2.24
\prob{24} Show that if $(x_n)$, $(y_n)$, $(z_n)$ are convergent sequences, then the sequence $(w_n)$ defined by $w_n := \text{mid}\{x_n, y_n, z_n\}$ is also convergent.
\pf
\end{document}
