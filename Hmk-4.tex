%%%%%%%%%%%%%%%%%%%%%%%%%%%%%%%%%%%%%%%%%
% Short Sectioned Assignment
% LaTeX Template
% Version 1.0 (5/5/12)
%
% This template has been downloaded from:
% http://www.LaTeXTemplates.com
%
% Original author:
% Frits Wenneker (http://www.howtotex.com)
%
% License:
% CC BY-NC-SA 3.0 (http://creativecommons.org/licenses/by-nc-sa/3.0/)
%
%%%%%%%%%%%%%%%%%%%%%%%%%%%%%%%%%%%%%%%%%

%----------------------------------------------------------------------------------------
%	PACKAGES AND OTHER DOCUMENT CONFIGURATIONS
%----------------------------------------------------------------------------------------

\documentclass[paper=a4, fontsize=11pt]{scrartcl} % A4 paper and 11pt font size

\usepackage[T1]{fontenc} % Use 8-bit encoding that has 256 glyphs
\usepackage{fourier} % Use the Adobe Utopia font for the document - comment this line to return to the LaTeX default
\usepackage[english]{babel} % English language/hyphenation
\usepackage{amsmath,amsfonts,amsthm} % Math packages

\usepackage{lipsum} % Used for inserting dummy 'Lorem ipsum' text into the template

\usepackage{sectsty} % Allows customizing section commands
\allsectionsfont{\centering \normalfont\scshape} % Make all sections centered, the default font and small caps

\usepackage{fancyhdr} % Custom headers and footers
\pagestyle{fancyplain} % Makes all pages in the document conform to the custom headers and footers
\fancyhead{} % No page header - if you want one, create it in the same way as the footers below
\fancyfoot[L]{} % Empty left footer
\fancyfoot[C]{} % Empty center footer
\fancyfoot[R]{\thepage} % Page numbering for right footer
\renewcommand{\headrulewidth}{0pt} % Remove header underlines
\renewcommand{\footrulewidth}{0pt} % Remove footer underlines
\setlength{\headheight}{13.6pt} % Customize the height of the header

\numberwithin{equation}{section} % Number equations within sections (i.e. 1.1, 1.2, 2.1, 2.2 instead of 1, 2, 3, 4)
\numberwithin{figure}{section} % Number figures within sections (i.e. 1.1, 1.2, 2.1, 2.2 instead of 1, 2, 3, 4)
\numberwithin{table}{section} % Number tables within sections (i.e. 1.1, 1.2, 2.1, 2.2 instead of 1, 2, 3, 4)

\setlength\parindent{0pt} % Removes all indentation from paragraphs - comment this line for an assignment with lots of text

%----------------------------------------------------------------------------------------
%	TITLE SECTION
%----------------------------------------------------------------------------------------

\newcommand{\horrule}[1]{\rule{\linewidth}{#1}} % Create horizontal rule command with 1 argument of height

\title{	
\normalfont \normalsize 
\textsc{University of Tennessee \\ Department of Mathematics} \\ [25pt] % Your university, school and/or department name(s)
\horrule{0.5pt} \\[0.4cm] % Thin top horizontal rule
\huge MATH 447 - Homework 4 \\ % The assignment title
\horrule{2pt} \\[0.5cm] % Thick bottom horizontal rule
}

\author{Robert D. French} % Your name

\date{\normalsize\today} % Today's date or a custom date

\begin{document}

\maketitle % Print the title

%----------------------------------------------------------------------------------------
%	PROBLEM 1
%----------------------------------------------------------------------------------------

\section{Uncountability of the Real Numbers}

The set $\mathbb{R}$ of real numbers is uncountable.

\paragraph{Proof.}

Assume the set $I=[0,1]$ is countable. This implies its elements are enumerable. That is, $I=\{x_1,x_2,\ldots,x_n,\ldots\}$.\\

Now construct the set $I_1=\{x \in I | x \neq x_1\}$ so that $I_1 \subset I$ and $ I_1 = I \setminus \{x_1\}$. Clearly this construction can progress in a recursive fashion so that $I_j = I_{j-1} \setminus \{x_j\}$, and $I_j \supseteq I_{j+1} \supseteq \cdots$.\\

Thus, we have establshed a collection of nested intervals, so we know $\exists \xi \in I_n \forall n \in \mathbb{N}$. Suppose $\xi = x_k$ for some $k \in \mathbb{N}$, then $\xi \notin I_k$, which contradicts the Nested Intervals Property. Clearly this is absurd, so we refute our hypothesis that $I=[0,1]$ is countable.\\

Since $[0,1] \subseteq \mathbb{R}$ is uncountable, we know that $\mathbb{R}$ itself is uncountable.$\blacksquare$

%----------------------------------------------------------------------------------------
%	PROBLEM 2
%----------------------------------------------------------------------------------------

\section{Binary representation of Real Numbers}

If $x \in [0,1]$, then there exists a sequence $(a_n)$ of 0s and 1s such that

\begin{align} 
\frac{a_1}{2} + \frac{a_2}{2^2} + \dots + \frac{a_n}{2^n} \leq x \leq \frac{a_1}{2} + \frac{a_2}{2^2} + \dots + \frac{a_n + 1}{2^n}
\end{align}

for all $n \in \mathbb{N}$. Conversely, each sequence of 0s and 1s is the binary representation of a unique real number in $[0,1]$.\\

\paragraph{Proof.} We begin by discussing an algorithm for constructing a sequence $(a_n)$ based on the choice $x$. We then demonstrate that the upper and lower bounds given above form a nested sequence of intervals, and since we know (by the Nested Intervals Property) that the intersection of such objects is nonempty, we can be confident that such a construction for $x$ is valid.\\

If $x \neq \frac{1}{2}$ belongs to the left subinterval $[0,\frac{1}{2}]$ we take $a_1 = 0$, while if $x$ belongs to the right subinterval $[\frac{1}{2},1]$ we take $a_1 = 1$. If $x = \frac{1}{2}$, then we may take $a_1$ to be either 0 or 1. In any case we have

\begin{align}
\frac{a_1}{2} \leq x \leq \frac{a_1}{2} + \frac{a_2 + 1}{2^2}.
\end{align}

We continue this bisection procedure, assigning at the $n$th stage the value $a_n = 0$ if $x$ is not the bisection point and lies in the left subinterval, and assigning the value $a_n = 1$ if $x$ lies in the right subinterval. Thus, we have a well-defined sequence $(a_n)$ of 0s and 1s such that Eqn 2.1 above holds.\\

Now we proceed to show that the upper and lower bounds given in Eqn 2.1 form a sequence of nested intervals. Let $L_n$ be the $n$th lower bound and $U_n$ the $n$th upper bound so that

\begin{align*}
&L_n := \sum_{1}^{n} \frac{a_i}{2^i}       &U_n := \sum_{1}^{n} \frac{a_i}{2^i} + \frac{1}{2^n}\\
&L_{n-1} := \sum_{1}^{n-1} \frac{a_i}{2^i} &U_{n-1} := \sum_{1}^{n-1} \frac{a_i}{2^i} + \frac{1}{2^{n-1}}
\end{align*}

which yields the following useful relationships

\begin{align*}
L_n &= L_{n-1} + \frac{a_n}{2^n} \\
U_n &= U_{n-1} - \frac{1}{2^{n-1}} + \frac{a_n}{2^n} + \frac{1}{2^n}
\end{align*}

We know that each $a_n$ is either 0 or 1, so we have two cases. If $a_n = 0$, then $\frac{a_n}{2^n} = 0$ so that $L_n = L_{n-1}$ and $U_n = U_{n-1} - \frac{1}{2^n} < U_{n-1}$. If $a_n = 1$, then $\frac{a_n}{2^n} = \frac{1}{2^n}$ so that $L_n = L_{n-1} + \frac{1}{2^n} > L_{n-1}$ and $U_n = U_{n-1} - \frac{1}{2^{n-1}} + \frac{2}{2^n} = U_{n-1}$. Thus, we have that $L_n \geq L_{n-1}$ and $U_n \leq U_{n-1}$. Therefore, for each $n \in \mathbb{N}$, we have

\begin{equation}
L_{n-1} \leq L_n < U_n \leq U_{n-1}
\end{equation}

so that the interval $[L_{n-1},U_{n-1}]$ contains the interval $[L_n,U_n]$.$\blacksquare$\\

%----------------------------------------------------------------------------------------
%	PROBLEM 3
%----------------------------------------------------------------------------------------

\section{Some problems concerning Intervals}

\subsection*{Exercise 2.5.3}

If $S \subseteq \mathbb{R}$ is a nonempty bounded set, and $I_S := [\inf S, \sup S]$, then $S \subseteq I_S$. Moreover, if $J$ is any closed bounded interval containing $S$, then $I_S \subseteq J$.\\

\paragraph{Proof.} Since $\inf S \leq s \forall s \in S$, and $\sup S \geq s \forall s \in S$, we know

\begin{equation}
\inf S \leq s \leq \sup S \forall s \in S
\end{equation}

Thus, $s \in [\inf S, \sup S] \forall s \in S$, so $S \subseteq I_S$.\\

Further, since $J$ is a closed bounded interval containing $S$ by hypothesis, we know that $\inf J \leq s \forall s \in S$ so that $\inf J$ is a lower bound for $S$, and we know that $\sup J \geq s \forall s \in S$ so that $\sup J$ is an upper bound for $S$. By definition of the infimum and supremum of $S$ we have that $\inf S \geq \inf J$ and $\sup S \leq \sup J$. Thus

\begin{equation}
\inf J \leq \inf S \leq \sup S \leq \sup J
\end{equation}

which confirms that $I_S \subseteq J$. $\blacksquare$\\

\subsection*{Exercise 2.5.10}

In the context of the proofs of Theorems 2.5.2 and 2.5.3, we have $\eta \in \cap_{n=1}^{\infty} I_n$. Also, $[\xi, \eta] = \cap_{n=1}^{\infty} I_n$.\\

\paragraph{Proof.} Suppose $\eta \notin \cap_{n=1}^{\infty} I_n$. Then $\exists m | \eta \notin I_m$. By definition, $\eta \leq b_m$, so we must have that $\eta \leq a_m$. But since

\begin{equation}
a_m \leq b_k \forall k \in \mathbb{N}
\end{equation}

then $a_m$ is a greater lower bound for $\{b_k | k \in \mathbb{N}\}$ than is $\eta$. This contradicts our hypothesis that $\eta = \inf \{b_n | n \in \mathbb{N}\}$. Thus, $\eta \in \cap_{n=1}^{\infty} I_n$.\\

We now show that $[\xi, \eta] = \cap_{n=1}^{\infty} I_n$. We begin by showing $[\xi, \eta] \subseteq \cap_{n=1}^{\infty} I_n$. Take $x \in [\xi, \eta]$, then $x \geq a_n \forall n$, since $x \geq \xi = \sup \{a_n | n \in \mathbb{N}\}$. Also, $x \leq b_n \forall n \in \mathbb{N}$, since $x \leq \eta = \inf \{b_n | n \in \mathbb{N}\}$. Thus,

\begin{equation}
a_n \leq x \leq b_n
\end{equation}

without regard to the choice of $n$. Thus, $x \in I_n \forall n \in \mathbb{N}$, so $x \in \cap_{n=1}^{\infty} I_n$. Thus, $[\xi, \eta] \subseteq \cap_{n=1}^{\infty} I_n$.

\newcommand{\xieta}{[\xi,\eta]}
\newcommand{\capin}{\cap_{n=1}^{\infty} I_n}
\newcommand{\fanin}{\forall n \in \mathbb{N}}
In order to establish that $\xieta = \capin$, we are now obliged to show that $\capin \subseteq \xieta$. Take $x \in \capin$, we know that $x \in [a_n, b_n] \fanin$, which means

\begin{equation}
a_n \leq x \leq b_n \fanin
\end{equation}

Thus $x$ is an upper bound for the set $\{a_n | n \in \mathbb{N}\}$ and a lower bound for $\{b_n | n \in \mathbb{N}\}$. This tells us that

\begin{equation}
a_n \leq \xi \leq x \leq \eta \leq b_n \fanin
\end{equation}

so that $x \in \xieta$. $\blacksquare$\\

\section{Some problems concerning Sequences and Their Limits}

\subsection*{Exercise 3.1.4}

For any $b \in \mathbb{R}$, $\lim(b/n) = 0$. 

\paragraph{Proof.} Consider first that, $\fanin$, $|b/n| \leq |b/n|$, and since $n > 0$, we can even say that $|b/n - 0| \leq |b| \cdot 1/n$. We can leverage the fact that $\lim(1/n)=0$ together with Theorem 3.1.10 (by taking $C=|b|$ and $m=1$) to conclude that $\lim(b/n)=0$ for any $b \in \mathbb{R}$. $\blacksquare$\\

\subsection*{Exercise 3.1.5}

Use the definition of the limit of a sequence to establish the following limits:

\subsubsection*{Part a}

$\lim(\frac{n}{n^2 + 1}) = 0$

\newcommand{\kep}{K(\epsilon)}
\paragraph{Proof.} Choose $\epsilon > 0$, the by the Archimedean Property, $\exists \kep$ such that $\frac{1}{\kep} < \epsilon$. Clearly,

\begin{equation}
\frac{1}{\kep + \frac{1}{\kep}} < \frac{1}{\kep} < \epsilon
\end{equation}

Since $\kep \in \mathbb{N}$ by definition, we know it is nonzero, so we may employ the identity $1 = \kep/\kep$ as follows

\begin{gather*}
1\cdot\frac{1}{\kep + \frac{1}{\kep}} < \frac{1}{\kep} < \epsilon \\
\left(\frac{\kep}{\kep}\right)\cdot\frac{1}{\kep + \frac{1}{\kep}} < \frac{1}{\kep} < \epsilon \\
\frac{\kep}{(\kep)^2 + 1} < \frac{1}{\kep} < \epsilon \\
\frac{\kep}{(\kep)^2 + 1} < \epsilon \\
\end{gather*}

We are guaranteed that $\frac{n}{n^2 + 1}$ is positive, so 

\begin{gather*}
\frac{\kep}{(\kep)^2 + 1} < \epsilon \\
\left|\frac{\kep}{(\kep)^2 + 1} - 0\right| < \epsilon \\
\end{gather*}

so that $\lim(\frac{n}{n^2 + 1}) = 0$. $\blacksquare$\\
\subsubsection*{Part d}
\paragraph{Proof.} Argument. $\blacksquare$\\

\subsection*{Exercise 3.1.6}
\subsubsection*{Part c}
\paragraph{Proof.} Argument. $\blacksquare$\\
\subsubsection*{Part d}
\paragraph{Proof.} Argument. $\blacksquare$\\

\subsection*{Exercise 3.1.8}
\paragraph{Proof.} Argument. $\blacksquare$\\
\end{document}
